\section{Introduction}
  The packet-scale congestion control is a newly proposed framework for 
  congestion control that promises to scale up the state of the art in 
  ultra-high speed congestion control by several orders of magnitude 
  \cite{Konda-infocom09}. This paradignm uses short probing-streams to probe 
  for available bandwidth at fine timescales by carefully controlling the 
  inter-packet spacings.

  The excellent adaptability of the paradigm to dynamically varying 
  bandwidth raises concern about its sensitivity to transient short-scale 
  burstiness in the cross traffic encountered on congested links. Thus, in 
  this paper, we continue the work done in \cite{Lovewell2011-Noise-TR} where 
  a simple periodic on-off model is used for the cross traffic. We extend the 
  analysis done in \cite{Lovewell2011-Noise-TR} by considering two scenarios.
  In one scenario, the cross traffic parameters are held constant and the RTT 
  of the Rapid flow is treated as a random variable. In the second 
  scenario, the RTT of the Rapid flow is held constant throughout the 
  connection and the cross traffic parameters are treated as random variables. 
  It is important to consider random RTTs and random cross traffic parameters 
  because it better approximates the scenarios found in real networks.

  In the rest of this paper, we develop the analysis for large-scale and 
  small-scale burts in sections 2 and 3 respectively. In section 4 we use 
  the derived equations for the Rapid throughput to study the performace of 
  the paradigm under different conditions and settings when the RTT is treated 
  as a uniform random variable. In section 5, we treat the parameters that 
  define the on-off cross traffic as random variables and study the 
  performance of the parameters. We conclude in section 6.
